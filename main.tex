\documentclass{article}
\usepackage[latin1]{inputenc}
\usepackage{enumitem}
\setlength{\parindent}{0pt}

\title{GLOSSARIO SWE}
\author{Marco Favaro}
\date{Anno accademico 2018-2019}


\begin{document}
 
\maketitle
\centerline{Glossario dei termini valevole per l'anno accademico 2018-2019.}
 

\newpage

\textbf{Protocollo: }accordo preso tra due o pi\'u parti

\bigbreak
\textbf{Progetto: }
    \begin{itemize}
        \item insieme di attivit\'a e compiti che:
        \begin{enumerate}
            \item devono raggiungere determinati obbiettivi con specifiche fissate
            \item ha delle date fissate
            \item ha a disposizione limitate risorse
            \item ha un consumo di risorse nello svolgimento
        \end{enumerate}
        
        \item Pianificazione : gestione risorse e responsabilit\'a
        \item Anilisi requisiti : cosa fare
        \item Progettazione : cosa fare (design)
        \item Realizzazione : fare ci\'o che serve perseguendo la \textbf{quantit\'a}
            \textbf{verificando} che non abbia errori
            \textbf{validando} i risultati rispetto le attese
    \end{itemize}

\bigbreak
\textbf{Disappointmet: }qualcosa di non atteso/aspettato (esempio una torta al posto di una pizza)

\bigbreak
\textbf{Engineering: }applicazioni della scienza e della matematica di principi noti e autorevoli (\textbf{best practice}), che spesso sono civili e sociali (responsabilit\'a e tecniche e professionali)

\bigbreak
\textbf{Software engineering: }disciplina per realizzare prodotti sw cos\'i impegnativi da richiedere attivit\'a collaborative.
Lungo l'intero ciclo di vita dello sviluppo \'e necessario:
    \begin{itemize}
        \item avere capacit\'a di produrre "in grande e in piccolo"
        \item garantire \textbf{efficacia}
        \item contenendo costi e risorse (\textbf{efficenza})
    \end{itemize}

\bigbreak
\textbf{Efficacia: }capacit\'a di raggiungere l'obiettivo

\bigbreak
\textbf{Efficenza: }abilit\'a di raggiungere l'obiettivo impegnando le risorse minime indispensabili

\bigbreak
\textbf{Prodotto SW: }
    \begin{itemize}
        \item Commessa: "atto di comprare", eseguito dal committente.
        \item Pacchetto: aggregato sw, serve per essere replicato (ce ne sono tanti identici).
        \item Componente: fatto per essere aggiunto.
        \item Servizio: outsourcing, assistenza dall'esterno.
    \end{itemize}

\bigbreak
\textbf{Ciclo di vita: }tempo per tenere in "vita" un software 
    \begin{itemize}
        \item Start: lo penso.
        \item Stop: lo ritiro dal mercato.
    \end{itemize}
    
\bigbreak
\textbf{Best Practice: } modo con cui affronto le attivit\'a del ciclo di vita. \'E data dalla conoscenza ed \'e il modo migliore di fare le cose.

\bigbreak
\textbf{Manutenzione: }
    \begin{itemize}
        \item Correttiva: correzione di bug (lo fa il produttore), e solitamente non la stessa persona che ha creato l'errore.
        \item Adattiva: quando sono presenti cambiamenti importanti sulla realt\'a in cui \'e operativo il sistema.
        \item Evolutiva: cambia il modo di fare le cose
    \end{itemize}

\bigbreak
\textbf{Body of knowledge: }raccogliere, organizzare, e consolidare la conoscenza.

\bigbreak
\textbf{Approccio sistematico: }modo metodico e rigoroso di lavorare che usa e studia le best practice.

\bigbreak
\textbf{Approccio disciplinato: }segue le regole.

\bigbreak
\textbf{Approccio quantificabile: }ci\'o che facciamo deve essere soggetto a una misura, per determinare l'efficenza del lavoro svolto.

\bigbreak
\textbf{Zero-latency: }metodo di studio "ci lavoro subito".

\bigbreak
\textbf{Zero-laxity: }metodo di studio "ci lavoro alla fine" (vicino alla scadenza).

\bigbreak
\textbf{Progetto} formato da tre strati principali:
    \begin{itemize}
        \item Customer: qualcuno che ha bisogno diuna cosa
            \begin{itemize}
                \item Oppurtunity: bisogno che vale la pena soddisfare.
                \item Stakeholder: quelli che giudicano la tua soluzione all'opportunit\'a.
            \end{itemize}
        \item Solution: derivare dall'opportunity i bisogni
            \begin{itemize}
                \item Requirements: requisiti.
                \item Software system: la soluzione che soddisfa l'opportunity.
            \end{itemize}
        \item Endeavor: sforzo importante per la realizzazione (attivit\'a svolta in team)
            \begin{itemize}
                \item Work: le cose che devo fare.
                \item Team: svolge work.
            \end{itemize}
    \end{itemize}
\textbf{WAY OF WORKING: }basamento del metodo di lavoro del team == best practice

\newpage

\textbf{PROCESSI SOFTWARE}
\bigbreak
Il software \'e buono se \textbf{dura}.
\bigbreak
Ci\'o che \'e sotto manutenzione ha una storia:
    \begin{itemize}
        \item Memoria di ci\'o che funzionava o ha funzionato $\rightarrow$ \textbf{versionamento}.
    \end{itemize}
\textbf{Prodotto software: }insieme di parti separate
    \begin{itemize}
        \item Come le parti stanno insieme $\rightarrow$ \textbf{configurazione}.
        \item Controllo di configurazione: gestisce le parti.
    \end{itemize}

\bigbreak
\textbf{Processi di ciclo di vita:} specifica delle \textbf{attivit\'a} da svolgere per abilitare corrette transazioni nel ciclo di vita.

\bigbreak
\textbf{Ciclo di vita:} conoscere il ciclo di vita previsto aiuta a calcolare costi e tempi di sviluppo.
    \begin{itemize}
        \item \textbf{Macchina a stati:} descrive un algoritmo, 
        "racconta" in modo non ambiguo come le cose avvengono. L'avanzamento degli stati sancisce le attivit\'a svolte e la maturazione del SW.
    \end{itemize}
    
\textbf{Modelli di ciclo di vita:} diversi per le transazioni previste tra gli stati
    \begin{itemize}
        \item \underline{ITERAZIONE:} \'e un ciclo, faccio pi\'u volte la stessa cosa. (Non posso capire a che punto sono del ciclo).
        \item \underline{INCREMENTO:} tengo ci\'o che ho e aggiungo ci\'o che manca. (Ho una traccaibilit\'a della situazione attuale).
        \item \underline{RIUSO:} 
            \begin{itemize}
                \item Occasionale: copia-incolla $\rightarrow$ basso costo (opportunistico).
                \item Sistematico: per progetto $\rightarrow$ maggior costo.
            \end{itemize}
        \item \underline{PROTOTIPO:} bozza del progetto finale.
    \end{itemize}
    
\bigbreak
\textbf{PROCESSO: }insieme di attivit\'a \underline{correlate} e \underline{coese} che trasformano ingressi in uscite sotto determinate regole cosnsumando risorse.
    \begin{itemize}
        \item correlate: stesso contesto.
        \item coese: non manca niente, tutto \'e importante.
    \end{itemize}

\bigbreak
\textbf{Economicit\'a:} \'e datto dalla somma dell'EFFICACIA e dell'EFFICIENZA. 
    \begin{itemize}
        \item misura dell'efficenza \'e la produttivit\'a.
        \item misura dell'efficacia sono gli obiettivi soddisfatti.
    \end{itemize}

\newpage
\textbf{ISO/IEC 12207: }standard dei cicli di vita.\newline Processi:
    \begin{itemize}
        \item primary.
        \item supporting.
        \item organisational.
    \end{itemize}
    
\bigbreak
\textbf{Com'\'e fatto il processo? }
    \begin{itemize}
        \item Processo vero e proprio.
        \item Attivit\'a: cose che voglio fare per attivare il processo.
        \item Tasks: cose che devo fae per raggiungere gli obiettivi.
    \end{itemize}
    
\bigbreak
\textbf{Standard: }si usano come corpo iniziale e si adattano al processo da svolgere.

\end{document}